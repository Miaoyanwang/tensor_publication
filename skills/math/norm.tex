\documentclass[11pt]{article}

\usepackage{fancybox}


\usepackage{color}
\usepackage{url}
\usepackage[margin=1in]{geometry}

\setlength\parindent{0pt}
\renewcommand{\textfraction}{0.0}
\renewcommand{\topfraction}{1.0}
%\renewcommand{\textfloatsep}{5mm}

\usepackage{comment}
% Definitions of handy macros can go here
\usepackage{amsmath,amssymb,amsthm,bm,mathtools}
\usepackage{multirow}
\usepackage{dsfont,multirow,hyperref,setspace,natbib,enumerate}
%\usepackage{dsfont,multirow,hyperref,setspace,enumerate}
\hypersetup{colorlinks,linkcolor={blue},citecolor={blue},urlcolor={red}} 
\usepackage{algpseudocode,algorithm}
\algnewcommand\algorithmicinput{\textbf{Input:}}
\algnewcommand\algorithmicoutput{\textbf{Output:}}
\algnewcommand\INPUT{\item[\algorithmicinput]}
\algnewcommand\OUTPUT{\item[\algorithmicoutput]}

\mathtoolsset{showonlyrefs=true}



\theoremstyle{plain}
\newtheorem{thm}{Theorem}[section]
\newtheorem{lem}{Lemma}
\newtheorem{prop}{Proposition}
\newtheorem{pro}{Property}
\newtheorem{assumption}{Assumption}

\theoremstyle{definition}
\newtheorem{defn}{Definition}
\newtheorem{cor}{Corollary}
\newtheorem{example}{Example}
\newtheorem{rmk}{Remark}


\renewcommand{\thefigure}{{S\arabic{figure}}}%
\renewcommand{\thetable}{{S\arabic{table}}}%
\renewcommand{\figurename}{{Supplementary Figure}}    
\renewcommand{\tablename}{{Supplementary Table}}    
\setcounter{figure}{0}   
\setcounter{table}{0}  


\def\MLET{\hat \Theta_{\text{MLE}}}
\newcommand{\cmt}[1]{{\leavevmode\color{red}{#1}}}



\usepackage{dsfont}

\usepackage{multirow}

\DeclareMathOperator*{\minimize}{minimize}



\usepackage{mathtools}
\mathtoolsset{showonlyrefs}
\newcommand*{\KeepStyleUnderBrace}[1]{%f
  \mathop{%
    \mathchoice
    {\underbrace{\displaystyle#1}}%
    {\underbrace{\textstyle#1}}%
    {\underbrace{\scriptstyle#1}}%
    {\underbrace{\scriptscriptstyle#1}}%
  }\limits
}
\usepackage{xr}

\input macros.tex



\title{Matrix Norm}

\date{\today}
\author{%
Jiaxin Hu
}



\begin{document}

% Makes the title and author information appear.

\maketitle


% Abstracts are required.
\section{Frobenius Norm}
\begin{lem}[Frobenius norm of product of matrices] For arbitrary two matrices, $\mA \in \mathbb{R}^{m \times r}$ and $\mB \in \mathbb{R}^{r \times n}$, we have
\[ \Fnorm{\mA\mB} \leq \norm{\mA} \Fnorm{\mB} , \]
	where $\norm{\cdot}$ is the spectral norm of matrix and $\Fnorm{\cdot}$ is Frobenius norm of matrix.
\end{lem}

\begin{proof}
	First, let $\onorm{\cdot}$ denote the $l_2$ norm of vector. The spectral norm of matrix $\mA \in \mathbb{R}^{m \times r}$ is defined as:
	\[ \norm{\mA} = \max_{x \in \mathbb{R}^r, \onorm{x} \leq 1} \onorm{\mA x}.  \]
	Therefore, we have $\onorm{\mA x} \leq \norm{\mA} \onorm{x}$ for $\forall x \in \mathbb{R}^r $. Let $\mB = [b_1,\dots, b_n] \in \mathbb{R}^{r \times n}$, where $b_j \in \mathbb{R}^{r}, j \in [n]$ are the columns of $\mB$. Then we have
	\[ \Fnorm{\mA\mB}^2 = \sum_{j = 1}^n \onorm{\mA b_j}^2 \leq  \norm{\mA}^2\sum_{j = 1}^n \onorm{b_j}^2 = \norm{\mA}^2 \Fnorm{\mB}^2 . \]
	That implies 
	\[\Fnorm{\mA\mB} \leq  \norm{\mA} \Fnorm{\mB}. \]
\end{proof}


\end{document}